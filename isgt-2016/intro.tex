As power grids transition from a centralized distribution model to a distributed model, maintaining a stable grid requires fine grained knowledge of how distributed renewables are affecting the state of the grid. Monitoring PQ on a distributed generation smartgrid requires a consumer level distributed sensor network. How do we visualize power quality (PQ) on the power grid scale using data gathered from residential utility customers while maintaining their privacy?

To answer this question, we developed and deployed an open source hardware and software system with a flexible privacy model focused on consumer level monitoring across Oahu. 

Our system is able to display PQ at both the consumer level and the grid level. At its core is a visualization algorithm that uses a quadtree based map in order to display PQ data without sacrificing users' privacy.

In late 2014 we performed a pilot study to test the feasibility of our hardware and software system. Data collected during this study showed strong correlations between PV production and daily voltage trends. We demonstrated that a grid wide view allows us to determine if PQ events take place at the consumer level or at the grid level. We  able to maintain the privacy of our users by allowing them to choose a privacy profile that they were most comfortable with.

Finally, we could produce these boxes for under \$100 each. This is roughly an order of magnitude cheaper than other power quality monitoring equipment and is better suited for our applications.